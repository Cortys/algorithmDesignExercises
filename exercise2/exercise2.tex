\documentclass[12pt,a4paper,bibliography=totocnumbered,listof=totocnumbered]{scrartcl}
\usepackage[ngerman]{babel}
\usepackage[utf8]{inputenc}
\usepackage[T1]{fontenc}

\usepackage{alltt}
\usepackage{upquote}
\usepackage{enumitem}
\usepackage{amsmath}
\usepackage{amsfonts}
\usepackage{amssymb}
\usepackage{pifont}
\usepackage{graphicx}
\usepackage{geometry}
\usepackage{setspace}
\usepackage[right]{eurosym}
\usepackage[printonlyused]{acronym}
\usepackage{lmodern}
\usepackage{blindtext}
\usepackage{soul,xcolor}
\usepackage{listings}
\usepackage{multicol}
\usepackage{syntax}
\usepackage[all,defaultlines=4]{nowidow}
\usepackage{tikz}
\usepackage{marvosym}
\usepackage{mathtools}
\usepackage{booktabs}
\usepackage{algorithm}
\usepackage[noend]{algpseudocode}

\lstset{basicstyle=\footnotesize, captionpos=b, breaklines=true, showstringspaces=false, tabsize=2, frame=lines, numbers=left, numberstyle=\tiny, xleftmargin=2em, framexleftmargin=2em, escapechar=@}

\lstset{literate=%
  {Ö}{{\"O}}1
  {Ä}{{\"A}}1
  {Ü}{{\"U}}1
  {ß}{{\ss}}2
  {ü}{{\"u}}1
  {ä}{{\"a}}1
  {ö}{{\"o}}1
}

\makeatletter
\def\l@lstlisting#1#2{\@dottedtocline{1}{0em}{1em}{\hspace{1,5em} Lst. #1}{#2}}
\makeatother

\geometry{a4paper, top=27mm, left=30mm, right=20mm, bottom=35mm, headsep=10mm, footskip=12mm}

\def\code#1{\texttt{#1}}

\sethlcolor{yellow}

\newcommand*\circled[1]{\tikz[baseline= (char.base)]{
            \node[shape=circle,draw,inner sep=2pt] (char) {#1};}}

\renewcommand{\qed}{\hfill\blacksquare}

\renewcommand{\sfdefault}{ppl}

\renewcommand{\thesubsection}{\alph{subsection}}
\renewcommand*\labelitemi{\blacktriangleright}

% \renewcommand{\syntleft}{\normalfont\itshape}
\renewcommand{\syntright}{}

\makeatletter
\renewcommand*\env@matrix[1][*\c@MaxMatrixCols c]{%
  \hskip -\arraycolsep
  \let\@ifnextchar\new@ifnextchar
  \array{#1}}

\renewcommand\maketitle{
   \begin{center}
     {\LARGE\bfseries Algorithmenentwurf \@title\par\vspace{0.5em}}
     {\scshape\@author\break\@date}
   \end{center}
}
\makeatother

\author{Lukas Brandt: 7011823, Clemens Damke: 7011488, Lukas Giesel: 7011495}


\begin{document}

\title{HA 2}
\date{21. April 1016}

\maketitle

\section*{Aufgabe 3}

\subsection{Beweis}

\emph{zu zeigen:} ($\dagger$) Gleichung aus der Aufgabenstellung.

\begin{itemize}
	\item \emph{Falls:} $k = 1 \implies n = 2$
		\begin{equation*}
			\begin{aligned}
				A =& \begin{bmatrix}
					a_1_1 & a_1_2 \\
					a_2_1 & a_2_2
				\end{bmatrix}\ B = \begin{bmatrix}
					b_1_1 & b_1_2 \\
					b_2_1 & b_2_2
				\end{bmatrix} \\
				C = A \cdot B =& \begin{bmatrix}
					c_1_1 = a_1_1 \cdot b_1_1 + a_1_2 \cdot b_2_1 & c_1_2 = a_1_1 \cdot b_1_2 + a_1_2 \cdot b_2_2\\
					c_2_1 = a_2_1 \cdot b_1_1 + a_2_2 \cdot b_2_1 & c_2_2 = a_2_1 \cdot b_1_2 + a_2_2 \cdot b_2_2
				\end{bmatrix} \\
				\implies& \text{($\dagger$) für $k = 1$, wenn man $a_i_j, b_i_j, c_i_j$ als $(1 \times 1)$-Matrizen interpretiert.}
			\end{aligned}
		\end{equation*}
	\item \emph{Falls:} $k > 1 \implies n = 2^k$
	\begin{equation*}
		A \cdot B = C = \begin{bmatrix}
			C_1_1 & C_1_2 \\
			C_2_1 & C_2_2
		\end{bmatrix} = \begin{bmatrix}[ccc|ccc]
			c_1_1 & \cdots & c_1_{\frac{n}{2}} & c_1_{\frac{n}{2} + 1} & \cdots & c_1_n \\
			\vdots & \ddots & \vdots & \vdots & \ddots & \vdots \\
			c_{\frac{n}{2}}_1 & \cdots & c_{\frac{n}{2}}_{\frac{n}{2}} & c_{\frac{n}{2}}_{\frac{n}{2} + 1} & \cdots & c_{\frac{n}{2}}_n \\
			\cmidrule(lr){1-6}
			c_{\frac{n}{2} + 1}_1 & \cdots & c_{\frac{n}{2} + 1}_{\frac{n}{2}} & c_{\frac{n}{2} + 1}_{\frac{n}{2} + 1} & \cdots & c_{\frac{n}{2} + 1}_n \\
			\vdots & \ddots & \vdots & \vdots & \ddots & \vdots \\
			c_n_1 & \cdots & c_n_{\frac{n}{2}} & c_n_{\frac{n}{2} + 1} & \cdots & c_n_n
		\end{bmatrix}
	\end{equation*}
	\begin{equation*}
		\begin{aligned}
			\forall i, j \in \{1, \ldots, n\}:\ c_i_j
			= &\sum_{m = 1}^{n} a_i_m \cdot b_m_j
			= \sum_{m = 1}^{\frac{n}{2}} a_i_m \cdot b_m_j
			+ \sum_{m = \frac{n}{2} + 1}^{n} a_i_m \cdot b_m_j \\
			= &\begin{cases}
				(A_1_1 \cdot B_1_1)_i_j + (A_1_2 \cdot B_2_1)_i_j & i, j \le \frac{n}{2}\ (C_1_1) \\
				(A_1_1 \cdot B_1_2)_i_{, j - \frac{n}{2}} + (A_1_2 \cdot B_2_2)_i_{, j - \frac{n}{2}} & i \le \frac{n}{2}, j > \frac{n}{2}\ (C_1_2) \\
				(A_2_1 \cdot B_1_1)_{i - \frac{n}{2}, }_j + (A_2_2 \cdot B_2_1)_{i - \frac{n}{2}, }_j & i > \frac{n}{2}, j \le \frac{n}{2}\ (C_2_1) \\
				(A_2_1 \cdot B_1_2)_{i  - \frac{n}{2}, }_{j - \frac{n}{2}} + (A_2_2 \cdot B_2_2)_{i  - \frac{n}{2}, }_{j - \frac{n}{2}} & i, j > \frac{n}{2}\ (C_2_2)
			\end{cases} \\
			= &\begin{cases}
				(A_1_1 \cdot B_1_1 + A_1_2 \cdot B_2_1)_i_j & i, j \le \frac{n}{2}\ (C_1_1) \\
				(A_1_1 \cdot B_1_2 + A_1_2 \cdot B_2_2)_i_{, j - \frac{n}{2}} & i \le \frac{n}{2}, j > \frac{n}{2}\ (C_1_2) \\
				(A_2_1 \cdot B_1_1 + A_2_2 \cdot B_2_1)_{i - \frac{n}{2}, }_j & i > \frac{n}{2}, j \le \frac{n}{2}\ (C_2_1) \\
				(A_2_1 \cdot B_1_2 + A_2_2 \cdot B_2_2)_{i  - \frac{n}{2}, }_{j - \frac{n}{2}} & i, j > \frac{n}{2}\ (C_2_2)
			\end{cases} \\
			\implies & \text{($\dagger$), da die vier Fälle jeweils die Einträge genau} \\
			& \text{einer Submatrix $C_i_j$ von $C$ beschreiben.}
			\end{aligned}
	\end{equation*}
	\qed
\end{itemize}

Es werden pro Submatrix von $C$ zwei Multiplikationen und eine Addition benötigt. Zur Multiplikation zweier Matrizen müssen vier Submatrizen berechnet und anschließend zusammengefügt werden. Insgesamt werden also 8 Multiplikationen und 4 Additionen benötigt.

\subsection{Algorithmus}

\begin{algorithm}
	\caption{Matrizenmultiplikation}
	\begin{algorithmic}[1]
		\Require $\exists k \in \mathbb{N}_0, n := 2^k: A = \begin{bmatrix} a_1_1 & \cdots & a_1_n \\ \vdots & \ddots & \vdots \\ a_n_1 & \cdots & a_n_n \end{bmatrix} \land B = \begin{bmatrix} b_1_1 & \cdots & b_1_n \\ \vdots & \ddots & \vdots \\ b_n_1 & \cdots & b_n_n \end{bmatrix}$
		\Statex
		\Function{M-Multiply}{$A$, $B$}
			\If{$n = 1$}
				\State \Return $a_1_1 \cdot b_1_1$ \Comment $\mathcal{O}(1)$
			\EndIf
			\Statex
			\State $A_1_1, A_1_2, A_2_1, A_2_2 \gets (\frac{n}{2} \times \frac{n}{2})\text{-Submatrizen von A}$ \Comment $\mathcal{O}(n^2)$
			\State $B_1_1, B_1_2, B_2_1, B_2_2 \gets (\frac{n}{2} \times \frac{n}{2})\text{-Submatrizen von B}$ \Comment $\mathcal{O}(n^2)$
			\Statex
			\ForAll{$i, j \in \{1, 2\}$}
				\State $T_1 \gets \Call{M-Multiply}{A_i_1, B_1_j}$ \Comment 8 Aufrufe pro \Call{M-Multiply}{}-Call.
				\State $T_2 \gets \Call{M-Multiply}{A_i_2, B_2_j}$ \Comment 8 Aufrufe pro \Call{M-Multiply}{}-Call.
				\State $C_i_j \gets \Call{M-Add}{T_1, T_2}$ \Comment $\mathcal{O}(n^2)$
			\EndFor
			\Statex
			\State \Return $\begin{bmatrix}
				C_1_1 & C_1_2 \\
				C_2_1 & C_2_2
			\end{bmatrix}$ \Comment $\mathcal{O}(n^2)$
		\EndFunction
		\Statex
		\Function{M-Add}{$A$, $B$}
			\ForAll{$i, j \in \{1, \ldots, n\}$}
				\State $c_i_j \gets a_i_j + b_i_j$
			\EndFor
			\Statex
			\State \Return $\begin{bmatrix} c_1_1 & \cdots & c_1_n \\ \vdots & \ddots & \vdots \\ c_n_1 & \cdots & c_n_n \end{bmatrix}$
		\EndFunction
	\end{algorithmic}
\end{algorithm}

\textbf{Laufzeit:}
\begin{equation*}
	\begin{align}
		T(n) &= 8c \cdot T(\frac{n}{2}) + c \cdot n^2 \\
		\underset{\mathrm{Mastertheorem}\ a > b}{\implies} T(n) &\le \frac{8}{8 - 2}c \cdot n^{\log_2{8}} = \frac{4}{3}c \cdot n^3 = \mathcal{O}(n^3)
	\end{align}
\end{equation*}

\end{document}
