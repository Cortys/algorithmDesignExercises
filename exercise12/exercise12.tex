\documentclass[12pt,a4paper,bibliography=totocnumbered,listof=totocnumbered]{scrartcl}
\usepackage[ngerman]{babel}
\usepackage[utf8]{inputenc}
\usepackage[T1]{fontenc}

\usepackage{alltt}
\usepackage{upquote}
\usepackage{enumitem}
\usepackage{amsmath}
\usepackage{amsfonts}
\usepackage{amssymb}
\usepackage{pifont}
\usepackage{graphicx}
\usepackage{geometry}
\usepackage{setspace}
\usepackage[right]{eurosym}
\usepackage[printonlyused]{acronym}
\usepackage{lmodern}
\usepackage{blindtext}
\usepackage{soul,xcolor}
\usepackage{listings}
\usepackage{multicol}
\usepackage{syntax}
\usepackage[all,defaultlines=4]{nowidow}
\usepackage{tikz}
\usepackage{marvosym}
\usepackage{mathtools}
\usepackage{booktabs}
\usepackage{algorithm}
\usepackage[noend]{algpseudocode}

\lstset{basicstyle=\footnotesize, captionpos=b, breaklines=true, showstringspaces=false, tabsize=2, frame=lines, numbers=left, numberstyle=\tiny, xleftmargin=2em, framexleftmargin=2em, escapechar=@}

\lstset{literate=%
  {Ö}{{\"O}}1
  {Ä}{{\"A}}1
  {Ü}{{\"U}}1
  {ß}{{\ss}}2
  {ü}{{\"u}}1
  {ä}{{\"a}}1
  {ö}{{\"o}}1
}

\makeatletter
\def\l@lstlisting#1#2{\@dottedtocline{1}{0em}{1em}{\hspace{1,5em} Lst. #1}{#2}}
\makeatother

\geometry{a4paper, top=27mm, left=30mm, right=20mm, bottom=35mm, headsep=10mm, footskip=12mm}

\def\code#1{\texttt{#1}}

\sethlcolor{yellow}

\newcommand*\circled[1]{\tikz[baseline= (char.base)]{
            \node[shape=circle,draw,inner sep=2pt] (char) {#1};}}

\renewcommand{\qed}{\hfill\blacksquare}

\renewcommand{\sfdefault}{ppl}

\renewcommand{\thesubsection}{\alph{subsection}}
\renewcommand*\labelitemi{\blacktriangleright}

% \renewcommand{\syntleft}{\normalfont\itshape}
\renewcommand{\syntright}{}

\makeatletter
\renewcommand*\env@matrix[1][*\c@MaxMatrixCols c]{%
  \hskip -\arraycolsep
  \let\@ifnextchar\new@ifnextchar
  \array{#1}}

\renewcommand\maketitle{
   \begin{center}
     {\LARGE\bfseries Algorithmenentwurf \@title\par\vspace{0.5em}}
     {\scshape\@author\break\@date}
   \end{center}
}
\makeatother

\author{Lukas Brandt: 7011823, Clemens Damke: 7011488, Lukas Giesel: 7011495}


\begin{document}

\title{HA 12}
\date{14. Juli 2016}

\maketitle

\section*{Aufgabe 23}
\label{sec:Aufgabe 23}

\subsection{Algorithmus}
\label{sub:Algorithmus}

Für alle eingehenden Requests $r = (v, p)$ auf Seiten $p$ der Größe $D$ an Knoten $v \in \{ v_1, v_2 \}$:

\begin{enumerate}
	\item \begin{cases}
		\text{Bewege $p$ zu $v$.} & \textbf{falls $p$ nicht bei $v$} \\
		\text{NOP} & \textbf{sonst}
	\end{cases}
	\item Schicke $p$ von Knoten $v$ aus.
\end{enumerate}

\subsection{Competitiveness}
\label{sub:Competitiveness}

o. B. d. A. fragen alle Requests die selbe Seite $p$ der Größe $D$ an, da Requests auf unterschiedliche Seiten sich gegenseitig nicht beeinflussen.
\begin{align*}
	A :=&\ \text{Der in \circled{a} beschriebene online Algorithmus.} \\
	O :=&\ \text{Ein optimaler offline Page-Migration Algorithmus.} \\
	r :=&\ \text{Eingabesequenz von Requests} = (r_1, \dots, r_n) \text{ mit } r_i = (t_i, p) \\
	t :=&\ (t_1, \dots, t_n) \\
	c_X(t) :=&\ \text{Von Algorihmus $X$ produzierte Kosten bei Knoteneingabefolge $t$.} \\
	b(i) :=&\ \min \{ j \in \{ \, b(i - 1) + 1, \dots, n \, \} \ |\ t_{j - 1} \neq t_j \}, \quad b(1) := 1 \\
	s(i) :=&\ (t_{b(2i - 1)}, \dots, t_{b(2i + 1) - 1}) \\
	|s|\, =&\ \text{Anzahl von Werten $i \in \mathbb{N}$ für die $s(i)$ definiert ist.}
	\intertext{$t$ (und somit $r$) wird also in Teilsequenzen $s(1), \dots, s(|s|)$ zerlegt, die jeweils Folgen von Knoten der Form $(v_x, \dots, v_x, v_y, \dots, v_y)$ mit $x, y \in \{ 1, 2 \} \land x \neq y$ sind.}
	\implies& \text{$c_O(s(i)) \ge d(v_1, v_2) = 1$, da entweder $v_1$ oder $v_2$ die Seite $p$ nicht haben kann.} \\
	\land\ & \text{$c_A(s(i)) \le 2 D$, da je nach Position von $p$ beim ersten Request an $v_x$ eine oder} \\
	& \text{ keine Page-Migration stattfindet (kostet max. $D \cdot d(v_1, v_2) = D$) und beim ersten} \\
	& \text{ Request an $v_y$ genau eine Page-Migration stattfindet (kostet $D$).} \\
	\implies& |s| \le c_O(t) \le c_A(t) \le |s| \cdot 2D \\
	\implies& \text{$A$ ist $2D$-competitive.}
\end{align*}
\qed\

\end{document}
