\documentclass[12pt,a4paper,bibliography=totocnumbered,listof=totocnumbered]{scrartcl}
\usepackage[ngerman]{babel}
\usepackage[utf8]{inputenc}
\usepackage[T1]{fontenc}

\usepackage{alltt}
\usepackage{upquote}
\usepackage{enumitem}
\usepackage{amsmath}
\usepackage{amsfonts}
\usepackage{amssymb}
\usepackage{pifont}
\usepackage{graphicx}
\usepackage{geometry}
\usepackage{setspace}
\usepackage[right]{eurosym}
\usepackage[printonlyused]{acronym}
\usepackage{lmodern}
\usepackage{blindtext}
\usepackage{soul,xcolor}
\usepackage{listings}
\usepackage{multicol}
\usepackage{syntax}
\usepackage[all,defaultlines=4]{nowidow}
\usepackage{tikz}
\usepackage{marvosym}
\usepackage{mathtools}
\usepackage{booktabs}
\usepackage{algorithm}
\usepackage[noend]{algpseudocode}

\lstset{basicstyle=\footnotesize, captionpos=b, breaklines=true, showstringspaces=false, tabsize=2, frame=lines, numbers=left, numberstyle=\tiny, xleftmargin=2em, framexleftmargin=2em, escapechar=@}

\lstset{literate=%
  {Ö}{{\"O}}1
  {Ä}{{\"A}}1
  {Ü}{{\"U}}1
  {ß}{{\ss}}2
  {ü}{{\"u}}1
  {ä}{{\"a}}1
  {ö}{{\"o}}1
}

\makeatletter
\def\l@lstlisting#1#2{\@dottedtocline{1}{0em}{1em}{\hspace{1,5em} Lst. #1}{#2}}
\makeatother

\geometry{a4paper, top=27mm, left=30mm, right=20mm, bottom=35mm, headsep=10mm, footskip=12mm}

\def\code#1{\texttt{#1}}

\sethlcolor{yellow}

\newcommand*\circled[1]{\tikz[baseline= (char.base)]{
            \node[shape=circle,draw,inner sep=2pt] (char) {#1};}}

\renewcommand{\qed}{\hfill\blacksquare}

\renewcommand{\sfdefault}{ppl}

\renewcommand{\thesubsection}{\alph{subsection}}
\renewcommand*\labelitemi{\blacktriangleright}

% \renewcommand{\syntleft}{\normalfont\itshape}
\renewcommand{\syntright}{}

\makeatletter
\renewcommand*\env@matrix[1][*\c@MaxMatrixCols c]{%
  \hskip -\arraycolsep
  \let\@ifnextchar\new@ifnextchar
  \array{#1}}

\renewcommand\maketitle{
   \begin{center}
     {\LARGE\bfseries Algorithmenentwurf \@title\par\vspace{0.5em}}
     {\scshape\@author\break\@date}
   \end{center}
}
\makeatother

\author{Lukas Brandt: 7011823, Clemens Damke: 7011488, Lukas Giesel: 7011495}


\begin{document}

\title{HA 1}
\date{14. April 1016}

\maketitle

\section*{Aufgabe 1}

\emph{zu zeigen:} ($\dagger$) Theorem aus Aufgabenstellung.

\begin{itemize}
	\item \emph{Induktionsanfang:} $n = 1$\\
	$\implies T(n) = T(1) \le c$\\
	Falls:
	\begin{itemize}
		\item \textbf{a < b:} $T(1) \le c < c \cdot \underbrace{\frac{b}{b - a}}_{> 1} \cdot 1$ \checkmark
		\item \textbf{a = b:} $T(1) \le c = c \cdot 1 \cdot (\underbrace{\log_b{1}}_{= 0} + 1)$ \checkmark
		\item \textbf{a > b:} $T(1) \le c < c \cdot \underbrace{\frac{b}{a - b}}_{> 1} \cdot \underbrace{1^\log_b{a}}_{= 1$ \checkmark
	\end{itemize}
	\item \emph{Induktionsvoraussetzung:} ($\dagger$) gilt für ein beliebiges aber festes $n$.
	\item \emph{Induktionsschritt:} $n \to n \cdot b$ (nächste Potenz von $b$)\\
	per Definition:
	\begin{equation*}
		T(nb) \le a \cdot T(\frac{nb}{b}) + c \cdot nb = a \cdot T(n) + c \cdot nb
	\end{equation*}
	Falls:
	\begin{itemize}
		\item \textbf{a < b:}
		\begin{equation*}
			\begin{align}
				T(nb) &\le a \cdot T(n) + c \cdot nb\\
				&\underset{\mathrm{I.V.}}{\le} a \cdot (c \cdot \frac{b}{b - a} \cdot n) + c \cdot nb\\
				&= (c \cdot nb \cdot \frac{a}{b - a}) + c \cdot nb = c \cdot nb \cdot (\frac{a}{b - a} + 1)\\
				&= c \cdot \frac{b}{b - a} \cdot nb
			\end{align}
		\end{equation*}
		\item \textbf{a = b:}
		\begin{equation*}
			\begin{align}
				T(nb) &\le a \cdot T(n) + c \cdot nb\\
				&\underset{\mathrm{I.V.}}{\le} a \cdot (cn \cdot (\log_{b}{n} + 1)) + c \cdot nb\\
				&\underset{\mathrm{a = b}}{=} b \cdot cn \cdot (\log_b{n} + 1) + c \cdot nb = c \cdot nb \cdot (\log_b{(nb)} + 1)
			\end{align}
		\end{equation*}
		\item \textbf{a > b:}
		\begin{equation*}
			\begin{align}
				T(nb) &\le a \cdot T(n) + c \cdot nb\\
				&\underset{\mathrm{I.V.}}{\le} a \cdot (c \cdot \frac{a}{a - b} \cdot n^{\log_b{a}}) + c \cdot nb\\
				&= ac \cdot \frac{a}{a - b} \cdot n^{\log_b{a}} + c \cdot nb\\
				&= b^{\log_b{a}}c \cdot \frac{a}{a - b} \cdot n^{\log_b{a}} + c \cdot nb\\
				&= c \cdot \frac{a}{a - b} \cdot (nb)^{\log_b{a}} + c \cdot nb\\
			\end{align}
		\end{equation*}
	\end{itemize}
\end{itemize}

\qed

\section*{Aufgabe 2}

\subsection{Funktionen ordnen}

\begin{equation*}
	\mathcal{O}(42) \underset{\mathrm{\circled{1}}}{\subset}
	\mathcal{O}(n^{10}) \underset{\mathrm{\circled{2}}}{\subset}
	\mathcal{O}(\sqrt{n}^{\log{n}}) \underset{\mathrm{\circled{3}}}{\subset}
	\mathcal{O}(5^{\frac{n}{2}}) \underset{\mathrm{\circled{4}}}{\subset}
	\mathcal{O}(3^{3n\log{n}})
\end{equation*}

\emph{Korrektheit:}

\begin{enumerate}[label=\protect\circled{\arabic*}]
	\item $\forall n > \sqrt[10]{42}:\ 42 < n^{10}$ \checkmark
	\item $\forall n > 2^{20}:\ 10 < \frac{1}{2}\log{n} \implies n^{10} < n^{\frac{1}{2}\log{n}} = \sqrt{n}^{\log{n}}$ \checkmark
	\item \begin{equation*}
			\begin{align}
				n \to \infty:\quad
				&\frac{1}{2}(\log{n})^2 < n\log{\sqrt{5}}\\
				\implies &\frac{1}{2}\log{n} < n \cdot \frac{\log{\sqrt{5}}}{\log{n}} = n \cdot \log_n{\sqrt{5}}\\
				\implies &n^{\frac{1}{2}\log{n}} < n^{n \cdot \log_n{\sqrt{5}}}\\
				\implies &\sqrt{n}^{\log{n}} < 5^{\frac{n}{2}}\quad\checkmark
			\end{align}
		\end{equation*}
	\item \begin{equation*}
		\begin{align}
			n \to \infty:\quad
			&\frac{1}{2} \cdot n < \log_{5}{(9)} \cdot n \log{n}\\
			\implies &5^{\frac{n}{2}} < 5^{\log_{5}{(9)} \cdot n \log{n}}\\
			\implies &5^{\frac{n}{2}} < 9^{n \log{n}} = 3^{3n\log{n}}\quad\checkmark
		\end{align}
	\end{equation*}
\end{enumerate}

\subsection{Laufzeiteigenschaften}

\begin{enumerate}
	\item Korrekt, denn: \begin{equation*}
		\begin{align}
			f(n) \in \Omega(g(n)) \implies &\exists c_0 > 0, n_0 > 0: \forall n \ge n_0:\ c_0 g(n) \le f(n)\\
			\implies &\exists \underbrace{c_1 > 0}_{\text{z. B.}\ c_1 = c_0^{-1}}, n_0 > 0: \forall n \ge n_0:\ g(n) \le c_1 f(n)\\
			\text{da}\ f(n) \in \mathbb{N}: f(n) \le f(n)^2 \implies &\exists c_1 > 0, n_0 > 0: \forall n \ge n_0:\ g(n) \le c_1 f(n)^2\\
			\implies &g(n) \in \mathcal{O}(f(n)^2)\quad\checkmark
		\end{align}
	\end{equation*}
	\item Falsch, denn:
		\begin{equation*}
			\begin{align*}
				\text{sei} f(n) &= n, &g(n) &= \log_2{n} &\implies f(n) \in \Omega(g(n))\\
				2^{f(n)} &= 2^n &2^{g(n)} &= 2^{log_2{n}} = n
			\end{align*}
		\end{equation*}
		\begin{equation*}
			\text{Angenommen}\ 2^{f(n)} \in \mathcal{O}(2^{g(n)}) \implies 2^n \in \mathcal{O}(n)\quad\text{\Lightning\ falsch}
		\end{equation*}

	\item Falsch, denn:
		\begin{equation*}
			\begin{align*}
				\text{sei} f(n) = n,\ g(n) = 1,\ h(n) = n &\implies f(n) \in \Omega(g(n))\\
				\text{Angenommen}\ f(n) \cdot h(n) \in \mathcal{O}(g(n) \cdot h(n)) &\implies n^2 \in \mathcal{O}(n)\quad\text{\Lightning\ falsch}
			\end{align*}
		\end{equation*}
\end{enumerate}

\end{document}
