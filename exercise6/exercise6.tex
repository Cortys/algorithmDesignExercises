\documentclass[12pt,a4paper,bibliography=totocnumbered,listof=totocnumbered]{scrartcl}
\usepackage[ngerman]{babel}
\usepackage[utf8]{inputenc}
\usepackage[T1]{fontenc}

\usepackage{alltt}
\usepackage{upquote}
\usepackage{enumitem}
\usepackage{amsmath}
\usepackage{amsfonts}
\usepackage{amssymb}
\usepackage{pifont}
\usepackage{graphicx}
\usepackage{geometry}
\usepackage{setspace}
\usepackage[right]{eurosym}
\usepackage[printonlyused]{acronym}
\usepackage{lmodern}
\usepackage{blindtext}
\usepackage{soul,xcolor}
\usepackage{listings}
\usepackage{multicol}
\usepackage{syntax}
\usepackage[all,defaultlines=4]{nowidow}
\usepackage{tikz}
\usepackage{marvosym}
\usepackage{mathtools}
\usepackage{booktabs}
\usepackage{algorithm}
\usepackage[noend]{algpseudocode}

\lstset{basicstyle=\footnotesize, captionpos=b, breaklines=true, showstringspaces=false, tabsize=2, frame=lines, numbers=left, numberstyle=\tiny, xleftmargin=2em, framexleftmargin=2em, escapechar=@}

\lstset{literate=%
  {Ö}{{\"O}}1
  {Ä}{{\"A}}1
  {Ü}{{\"U}}1
  {ß}{{\ss}}2
  {ü}{{\"u}}1
  {ä}{{\"a}}1
  {ö}{{\"o}}1
}

\makeatletter
\def\l@lstlisting#1#2{\@dottedtocline{1}{0em}{1em}{\hspace{1,5em} Lst. #1}{#2}}
\makeatother

\geometry{a4paper, top=27mm, left=30mm, right=20mm, bottom=35mm, headsep=10mm, footskip=12mm}

\def\code#1{\texttt{#1}}

\sethlcolor{yellow}

\newcommand*\circled[1]{\tikz[baseline= (char.base)]{
            \node[shape=circle,draw,inner sep=2pt] (char) {#1};}}

\renewcommand{\qed}{\hfill\blacksquare}

\renewcommand{\sfdefault}{ppl}

\renewcommand{\thesubsection}{\alph{subsection}}
\renewcommand*\labelitemi{\blacktriangleright}

% \renewcommand{\syntleft}{\normalfont\itshape}
\renewcommand{\syntright}{}

\makeatletter
\renewcommand*\env@matrix[1][*\c@MaxMatrixCols c]{%
  \hskip -\arraycolsep
  \let\@ifnextchar\new@ifnextchar
  \array{#1}}

\renewcommand\maketitle{
   \begin{center}
     {\LARGE\bfseries Algorithmenentwurf \@title\par\vspace{0.5em}}
     {\scshape\@author\break\@date}
   \end{center}
}
\makeatother

\author{Lukas Brandt: 7011823, Clemens Damke: 7011488, Lukas Giesel: 7011495}


\begin{document}

\title{HA 6}
\date{26. Mai 2016}

\maketitle

\section*{Aufgabe 9}
\label{sec:Aufgabe 9}

\subsection{Rekursive Definition}
\label{sub:Rekursive Definition}

\begin{align*}
	\tau(i, j) := &\ \mathbb{N}_{0} \cap \begin{cases}
		[i, j] & \textbf{if } i \le j \\
		[0, n - 1]\, \backslash\, ]j, i[ & \textbf{if } i > j
	\end{cases} \\
	t[i, j] := &\ \begin{cases}
		0 \hfill \textbf{if } \min \{ \, q\ |\ j \equiv i + q \text{ mod } n \, \} \le 1 \\
		\min \{ \, t[i, k] + \Delta(i, k, j) + t[k, j]\ |\ k \in \tau((i + 1 \text{ mod } n), (j - 1 \text{ mod } n))\, \} \quad \textbf{else}
	\end{cases}
\end{align*}

\subsection{Algorithmus}
\label{sub:Algorithmus}

\begin{algorithm}
	\caption{Minimales Gewicht Triangulierung Problem}

	\begin{algorithmic}[1]
		\Function{MinimalesGewichtTriangulierung}{$\langle v_0, \dots, v_{n - 1} \rangle$}
			\If{$n \le 2$}
				\Return{$0$} \Comment{$\mathcal{O}(1)$}
			\EndIf{}
			\Statex\
			\For{$i \gets 0 \ato n - 1$}  \Comment{$\mathcal{O}(n)$}
				\State{$t[i, (i + 1 \text{ mod } n)] \gets 0$}
			\EndFor\
			\Statex\
			\For{$l \gets 2 \ato n - 1$} \Comment{$\mathcal{O}(n)$}
				\For{$i \gets 0 \ato \begin{cases}
					\textbf{if } l \neq n - 1 \text{:} & n - 1 \\
					\textbf{else}\text{:} & 0
				\end{cases}$} \Comment{$\mathcal{O}(n) \cdot \mathcal{O}(n) = \mathcal{O}(n^2)$}
					\State{$m \gets \infty$}
					\State{$j \gets i + l \mod n$}
					\For{$d \gets 1 \ato l - 1 $} \Comment{$\mathcal{O}(n^2) \cdot \mathcal{O}(n) = \mathcal{O}(n^3)$}
						\State{$k \gets i + d \mod n$}
						\State{$m_0 \gets t[i, k] + |v_i v_k| + |v_k v_j| + |v_j v_i| + t[k, j]$}
						\If{$m_0 < m$}
							$m \gets m_0$
						\EndIf\
					\EndFor\
					\State{$t[i, j] \gets m$}
				\EndFor\
			\EndFor\
			\Statex\
			\State{\Return{$t[0, n - 1]$}}
		\EndFunction{}
	\end{algorithmic}
\end{algorithm}

\subsection{Laufzeit}
\label{sub:Laufzeit}

Der Algorithmus besteht im Wesentlichen aus drei verschachtelten Schleifen, die alle bis zu $n$ mal ausgeführt werden. Die Operationen in den Schleifen benötigen alle nur konstante Zeit. Insgesamt ergibt sich also eine Laufzeit von $\mathcal{O}(n^3)$.

\section*{Aufgabe 10}
\label{sec:Aufgabe 10}

\setcounter{subsection}{0}

\subsection{Rekursive Definition}
\label{sub:Rekursive Definition}

\begin{align*}
	D := &\ \{ d_1, \dots, d_n \} \\
	t[i, j] := &\ \begin{cases}
		0 & \textbf{if } i > \max D \\
		t[i + 1, \max \{ j - 1, 0 \}] & \textbf{if } i \le \max D \land (j > 0 \lor i \notin D) \\
		\min \{ \, c_{r} + t[i + 1, \ell_{r} - 1]\ |\ r \in \{ 1, \dots, k \} \, \} & \textbf{else}
	\end{cases}
\end{align*}

Optimale Lösung ($\hat{=}$ minimale Gesamtausgaben) ist $t[0, 0]$.

\subsection{Algorithmus}
\label{sub:Algorithmus}

\begin{algorithm}
	\caption{Parkschein Problem}
	\begin{algorithmic}[1]
		\Require{$D = \{ d_1, \dots, d_n \}$}
		\Require{$L = \{ \ell_1, \dots, \ell_k \}$}
		\Require{$C = \{ c_1, \dots, c_k \}$}
		\Statex\
		\Function{Parkschein}{$D, L, C$}
			\State{$m_d \gets \max D$} \Comment{$\mathcal{O}(|D|) = \mathcal{O}(n)$}
			\State{$m_\ell \gets \max L$} \Comment{$\mathcal{O}(|L|) = \mathcal{O}(k)$}
			\Statex\
			\For{$j \gets 0 \ato m_\ell - 1$} \Comment{$\mathcal{O}(m_\ell)$}
				\State{$t[m_d + 1, j] \gets 0$}
			\EndFor\
			\For{$i \gets m_d \ato 0$} \Comment{$\mathcal{O}(m_d)$}
				\For{$j \gets 1 \ato m_\ell - 1$} \Comment{$\mathcal{O}(m_d \cdot m_l)$}
					\State{$t[i, j] \gets t[i + 1, j - 1]$}
				\EndFor\
				\If{$i \in D$} \Comment{$\mathcal{O}(m_d \cdot 1)$ mit $D$ als Hashset}
					\State{$m \gets \infty$}
					\For{$r \gets 1 \ato k$} \Comment{$\mathcal{O}(m_d \cdot k)$}
						\State{$m_0 \gets c_r + t[i + 1, \ell_r - 1]$}
						\If{$m_0 < m$}
							$m \gets m_0$
						\EndIf\
					\EndFor\
					\State{$t[i, 0] \gets m$}
				\Else\
					\State{$t[i, 0] \gets t[i + 1, 0]$}
				\EndIf\
			\EndFor\
			\Statex\
			\State{\Return{$t[0, 0]$}}
		\EndFunction{}
	\end{algorithmic}
\end{algorithm}

\subsection{Laufzeit}
\label{sub:Laufzeit}

Der Algorithmus ist pseudopolynomiell, da $m_d = \mathcal{O}(2^n)$ und $m_\ell = \mathcal{O}(2^k)$. \\
Wegen Zeile 7 ergibt sich also insgesamt $\mathcal{O}(\max D \cdot \max L) = \mathcal{O}(2^{n + k})$.

\section*{Aufgabe 11}
\label{sec:Aufgabe 11}

\setcounter{subsection}{0}

\subsection{Rekursive Definition}
\label{sub:Rekursive Definition}

\begin{align*}
	\pi_i := &\ j, \text{sodass } \sigma_i = \nu_j \\
	p[i, s] := &\ \begin{cases}
		0 & \textbf{if } i > t \\
		\min \{ \, d(\nu_s, \sigma_i) + p[i + 1, s],\ D \cdot d(\nu_s, \sigma_i) + p[i + 1, \pi_i] \, \} & \textbf{else}
	\end{cases}
\end{align*}

Optimale Lösung ($\hat{=}$ minimale Kosten) ist $p[1, 1]$.

\subsection{Algorithmus}
\label{sub:Algorithmus}

\begin{algorithm}
	\caption{Page Migration Problem}
	\begin{algorithmic}[1]
		\Function{PageMigration}{$\langle \nu_1, \dots, \nu_n \rangle, \langle \sigma_1, \dots, \sigma_t \rangle, D, d$}
			\For{$s \gets 1 \ato n$} \Comment{$\mathcal{O}(n)$}
				\State{$p[t + 1, s] \gets 0$}
			\EndFor\
			\Statex\
			\For{$i \gets t \ato 1$} \Comment{$\mathcal{O}(t)$}
				\For{$j \gets 1 \ato n$} \Comment{$\mathcal{O}(t \cdot n)$}
					\If{$\sigma_i = \nu_j$}
						\State{$\pi_i \gets j$}
						\State{\textbf{break}}
					\EndIf\
				\EndFor\
				\Statex\
				\For{$s \gets 1 \ato n$} \Comment{$\mathcal{O}(t\cdot n)$}
					\State{$p_0 \gets  d(\nu_s, \sigma_i) + p[i + 1, s]$}
					\State{$p_1 \gets D \cdot d(\nu_s, \sigma_i) + p[i + 1, \pi_i]$}
					\State{$p[i, s] \gets \min \{ p_0, p_1 \}$} \Comment{$\mathcal{O}(t \cdot n \cdot 1)$}
				\EndFor\
			\EndFor\
			\Statex\
			\State{\Return{$p[1, 1]$}}
		\EndFunction{}
	\end{algorithmic}
\end{algorithm}

\subsection{Laufzeit}
\label{sub:Laufzeit}

Der Algorithmus besteht im Wesentlichen aus zwei verschachtelten Schleifen, die jeweils $t$ bzw. $n$ mal ausgeführt werden. Die Operationen in den Schleifen benötigen alle nur konstante Zeit. Insgesamt ergibt sich also eine Laufzeit von $\mathcal{O}(t \cdot n)$.

\end{document}
