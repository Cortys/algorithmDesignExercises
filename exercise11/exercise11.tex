\documentclass[12pt,a4paper,bibliography=totocnumbered,listof=totocnumbered]{scrartcl}
\usepackage[ngerman]{babel}
\usepackage[utf8]{inputenc}
\usepackage[T1]{fontenc}

\usepackage{alltt}
\usepackage{upquote}
\usepackage{enumitem}
\usepackage{amsmath}
\usepackage{amsfonts}
\usepackage{amssymb}
\usepackage{pifont}
\usepackage{graphicx}
\usepackage{geometry}
\usepackage{setspace}
\usepackage[right]{eurosym}
\usepackage[printonlyused]{acronym}
\usepackage{lmodern}
\usepackage{blindtext}
\usepackage{soul,xcolor}
\usepackage{listings}
\usepackage{multicol}
\usepackage{syntax}
\usepackage[all,defaultlines=4]{nowidow}
\usepackage{tikz}
\usepackage{marvosym}
\usepackage{mathtools}
\usepackage{booktabs}
\usepackage{algorithm}
\usepackage[noend]{algpseudocode}

\lstset{basicstyle=\footnotesize, captionpos=b, breaklines=true, showstringspaces=false, tabsize=2, frame=lines, numbers=left, numberstyle=\tiny, xleftmargin=2em, framexleftmargin=2em, escapechar=@}

\lstset{literate=%
  {Ö}{{\"O}}1
  {Ä}{{\"A}}1
  {Ü}{{\"U}}1
  {ß}{{\ss}}2
  {ü}{{\"u}}1
  {ä}{{\"a}}1
  {ö}{{\"o}}1
}

\makeatletter
\def\l@lstlisting#1#2{\@dottedtocline{1}{0em}{1em}{\hspace{1,5em} Lst. #1}{#2}}
\makeatother

\geometry{a4paper, top=27mm, left=30mm, right=20mm, bottom=35mm, headsep=10mm, footskip=12mm}

\def\code#1{\texttt{#1}}

\sethlcolor{yellow}

\newcommand*\circled[1]{\tikz[baseline= (char.base)]{
            \node[shape=circle,draw,inner sep=2pt] (char) {#1};}}

\renewcommand{\qed}{\hfill\blacksquare}

\renewcommand{\sfdefault}{ppl}

\renewcommand{\thesubsection}{\alph{subsection}}
\renewcommand*\labelitemi{\blacktriangleright}

% \renewcommand{\syntleft}{\normalfont\itshape}
\renewcommand{\syntright}{}

\makeatletter
\renewcommand*\env@matrix[1][*\c@MaxMatrixCols c]{%
  \hskip -\arraycolsep
  \let\@ifnextchar\new@ifnextchar
  \array{#1}}

\renewcommand\maketitle{
   \begin{center}
     {\LARGE\bfseries Algorithmenentwurf \@title\par\vspace{0.5em}}
     {\scshape\@author\break\@date}
   \end{center}
}
\makeatother

\author{Lukas Brandt: 7011823, Clemens Damke: 7011488, Lukas Giesel: 7011495}


\begin{document}

\title{HA 11}
\date{7. Juli 2016}

\maketitle

\section*{Aufgabe 23}
\label{sec:Aufgabe 23}

\subsection{ILP}
\label{sub:ILP}

\begin{align*}
	\textbf{minimiere } & c \cdot x \text{ für gegebenes $c = (c_1\ \cdots\ c_m)$} \\
	\textbf{s. t. } & A \cdot x \ge {\underbrace{(1\ \cdots\ 1)}_{m \text{-mal}}}^T,\ x \in \{ 0, 1 \}^m \\
	& A := \begin{pmatrix}
		a_{1 1} & \cdots & a_{1 m} \\
		\vdots & \ddots & \vdots \\
		a_{n 1} & \cdots & a_{n m}
	\end{pmatrix},\ a_{ij} := \begin{cases}
		1 & \textbf{falls } e_i \in S_j \\
		0 & \textbf{sonst}
	\end{cases}
\end{align*}
Minimale Lösungen aus $\{ 0, 1 \}^m$ sind identisch zu minimalen Lösungen aus $\mathbb{N}_0^m$, da letztere an keiner Stelle $x_i$ kleiner als erstere sein können und $\forall\, x_i > 1$ auch $x_i = 1$ eine valide Lösung ist, womit $x \notin \{ 0, 1 \}^m$ dann also nicht minimal wäre. $\implies$ obiges 0-1-LP $\equiv$ ILP.

\subsection{Approximationsalgorithmus}
\label{sub:Approximationsalgorithmus}

\begin{enumerate}
	\item Löse LP aus \circled{a} reelwertig. Lösung $x' = (x'_1\ \cdots\ x'_m)^T \in \mathbb{R}_{\ge 0}^m$
	\item Berechne Lösung $x := (x_1\ \cdots\ x_m)^T \in \{ 0, 1 \}^m,\ x_i := \begin{cases}
		1 & \textbf{falls } x'_i \ge \frac{1}{f} \\
		0 & \textbf{sonst}
	\end{cases}$
\end{enumerate}

\textbf{Korrektheit:}
\begin{align*}
	& \forall\, i \in \{ 1, \dots, n \}: \sum_{j = 1}^{m} a_{i j} \le f\ \land\ \sum_{j = 1}^{m} a_{i j} \cdot x'_{j} \ge 1 \\
	\implies & \forall\, i \in \{ 1, \dots, n \}: \exists\, j \in \{ 1, \dots, m \}: a_{i j} = 1\ \land\ x'_j \ge \frac{1}{f} \\
	\implies & \forall\, i \in \{ 1, \dots, n \}: \exists\, j \in \{ 1, \dots, m \}: a_{i j} = 1\ \land\ x_j = 1 \\
	\implies & A \cdot x \ge {\underbrace{(1\ \cdots\ 1)}_{m \text{-mal}}}^T
\end{align*}

\textbf{Approximationsfaktor:}
\begin{align*}
	& c \cdot x_0 \ge c \cdot x',\ x_0 := \text{optimale ganzzahlige Lösung} \\
	& \forall\, j \in \{ 1, \dots, m \}: x'_j \cdot f \ge x_j \\
	\implies & c \cdot x_0 \cdot f \ge c \cdot x' \cdot f \ge c \cdot x \\
	\implies & \text{$x$ ist max. $f$-mal schlechter als die optimale ganzzahlige Lösung.}
\end{align*}

\section*{Aufgabe 24}
\label{sec:Aufgabe 24}

\begin{align*}
	A := &\ \text{online Algorithmus aus Aufgabenstellung} \\
	O := &\ \text{optimaler Algorithums} \\
	s := &\ \text{Eingabesequenz von Gegenständen} = (g_1, \dots, g_n) \\
	w(g) := &\ \text{Gewicht des Gegenstands $g$} \\
	a := &\ \text{aufsteigende Zeitpunkte zu denen $A$ Koffer öffnet} = (a_1 = 1, \dots, a_k) \in \{ 1, \dots, n \}^k \\
	p(i) := &\ (g_\alpha, \dots, g_\beta) \text{ mit } \alpha := a_{2i - 1},\ \beta := a_{2i + 1} - 1 \\
	\implies & \forall\, i \in \mathbb{N}: p(i) \text{ enthält für zwei sukzessive Koffer alle eingepackten Gegenstände.} \\
	\implies & \forall\, p(i) = (g_\alpha, \dots, g_\beta), i \in \mathbb{N}: \sum_{j = \alpha}^{\beta} w(g_j) > 1 \text{, da sonst kein zweiter Koffer geöffnet würde.} \\
	\implies & \forall\, i \in \mathbb{N}: \text{Gegenstände in $p(i)$ verursachen in $O$ das Öffnen von min. einem Koffer.} \\
	\implies & \text{Anzahl von Koffern in $A$ ist max. doppelt so groß, wie in $O$.} \\
	\implies & \text{$A$ ist 2-competitive.}
\end{align*}

\end{document}
