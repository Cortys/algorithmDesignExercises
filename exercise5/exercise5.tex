\documentclass[12pt,a4paper,bibliography=totocnumbered,listof=totocnumbered]{scrartcl}
\usepackage[ngerman]{babel}
\usepackage[utf8]{inputenc}
\usepackage[T1]{fontenc}

\usepackage{alltt}
\usepackage{upquote}
\usepackage{enumitem}
\usepackage{amsmath}
\usepackage{amsfonts}
\usepackage{amssymb}
\usepackage{pifont}
\usepackage{graphicx}
\usepackage{geometry}
\usepackage{setspace}
\usepackage[right]{eurosym}
\usepackage[printonlyused]{acronym}
\usepackage{lmodern}
\usepackage{blindtext}
\usepackage{soul,xcolor}
\usepackage{listings}
\usepackage{multicol}
\usepackage{syntax}
\usepackage[all,defaultlines=4]{nowidow}
\usepackage{tikz}
\usepackage{marvosym}
\usepackage{mathtools}
\usepackage{booktabs}
\usepackage{algorithm}
\usepackage[noend]{algpseudocode}

\lstset{basicstyle=\footnotesize, captionpos=b, breaklines=true, showstringspaces=false, tabsize=2, frame=lines, numbers=left, numberstyle=\tiny, xleftmargin=2em, framexleftmargin=2em, escapechar=@}

\lstset{literate=%
  {Ö}{{\"O}}1
  {Ä}{{\"A}}1
  {Ü}{{\"U}}1
  {ß}{{\ss}}2
  {ü}{{\"u}}1
  {ä}{{\"a}}1
  {ö}{{\"o}}1
}

\makeatletter
\def\l@lstlisting#1#2{\@dottedtocline{1}{0em}{1em}{\hspace{1,5em} Lst. #1}{#2}}
\makeatother

\geometry{a4paper, top=27mm, left=30mm, right=20mm, bottom=35mm, headsep=10mm, footskip=12mm}

\def\code#1{\texttt{#1}}

\sethlcolor{yellow}

\newcommand*\circled[1]{\tikz[baseline= (char.base)]{
            \node[shape=circle,draw,inner sep=2pt] (char) {#1};}}

\renewcommand{\qed}{\hfill\blacksquare}

\renewcommand{\sfdefault}{ppl}

\renewcommand{\thesubsection}{\alph{subsection}}
\renewcommand*\labelitemi{\blacktriangleright}

% \renewcommand{\syntleft}{\normalfont\itshape}
\renewcommand{\syntright}{}

\makeatletter
\renewcommand*\env@matrix[1][*\c@MaxMatrixCols c]{%
  \hskip -\arraycolsep
  \let\@ifnextchar\new@ifnextchar
  \array{#1}}

\renewcommand\maketitle{
   \begin{center}
     {\LARGE\bfseries Algorithmenentwurf \@title\par\vspace{0.5em}}
     {\scshape\@author\break\@date}
   \end{center}
}
\makeatother

\author{Lukas Brandt: 7011823, Clemens Damke: 7011488, Lukas Giesel: 7011495}


\begin{document}

\title{HA 5}
\date{19. Mai 2016}

\maketitle

\section*{Aufgabe 9}
\label{sec:Aufgabe 9}

\subsection{Algorithmus}
\label{sub:Algorithmus}

\begin{algorithm}
	\caption{Maximales bipartites Matching}
	\begin{algorithmic}[1]
		\Require{$(L \cup R, E,w)$ ist bipartiter Graph mit linker Seite $L$, rechter Seite $R$ und Gewichtsfunktion $w: L \to \mathbb{R}$.}
		\Statex\
		\Function{MaxBipartiteMatching}{$L, R, E, w$}
			\State\ $U \gets \{ \}$ \Comment{$\mathcal{O}(1)$}
			\State\ $L \gets \Call{SortDescending}{L\ \text{by}\ w}$ \Comment{$\mathcal{O}(|L| \log{|L|})$}
			\ForAll{$l \in L$} \Comment{$\mathcal{O}(|L|)$}
				\If{\Call{IsMatching}{$U \cup \{ l \}, R, E$}} \Comment{$\mathcal{O}(|L| \cdot x)$}
					\State\ $U \gets U \cup \{ l \}$ \Comment{$\mathcal{O}(|L| \cdot 1)$}
				\EndIf\
			\EndFor\
			\State\ \Return{$U$}
		\EndFunction\
		\Statex\
		\Function{IsMatching}{$L, R, E$}
			\If{$|L| = 0$}
				\State \Return{true}
			\EndIf\

			\State\ $l \gets \Call{FirstElement}{L}$ \Comment{$\mathcal{O}(1)$}
			\ForAll{$r \in R$} \Comment{$\mathcal{O}(|R|)$}
				\If{$\{ l, r \} \in E \land \Call{IsMatching}{L \backslash \{l\}, R \backslash \{r\}, E}$} \Comment{$\mathcal{O}(|R|!)$}
					\State\ \Return{true}
				\EndIf\
			\EndFor\

			\State\ \Return{false}
		\EndFunction\
	\end{algorithmic}
\end{algorithm}

Nicht effizient.

\section{Aufgabe 10}
\label{sec:Aufgabe 10}

\begin{align*}
	M &:= (E = J, U) \\
	U &:= \{ \, B \subseteq E\ |\ \exists \, \text{Bijektion}\, \pi: B \to \{ 1, \dots, |B| \}: \, \forall j \in B: \, \pi(j) \le d_{j} \, \} \\
	p(B) &:= \sum_{j \in B} p_{j},\ \forall B \in U
	\intertext{Gesucht ist ein $B \in U$ mit maximalem $p(B)$, da alle Jobs in $B$ rechtzeitig geschafft werden und somit keine Strafe verursachen. Das Maximieren der nicht anfallenden Strafe minimiert somit die anfallende Strafe. Als Gewichtsfunktion kann also $w(j) := p_j, \forall j \in E$ verwendet werden.}
	\intertext{\emph{z. z.} $M$ ist Matroid:}
	\shortintertext{\mathbb{\circled{1}} $\emptyset \in U$. Per Definition von $U$ wahr.}
	\shortintertext{\mathbb{\circled{2}} $\forall \, A \subseteq B \in U:}
	\pi_{B} &:= \text{Eine, gemäß Definition aus $U$, zu $B$ passende Bijektion.} \\
	\pi_{A} &:= \pi_{B} \, \backslash \, \{ \, (j, \pi_{B}(j))\ |\ j \in B \, \backslash \, A \, \} \\
	&\text{Da $\pi_{A} \subseteq \pi_{B}$: $\forall j \in A: \, \pi_{A}(j) = \pi_{B}(j) \le d_{j} $} \\
	\implies & A \in U
	\shortintertext{\mathbb{\circled{3}} Austauscheigenschaft: $\forall A, B \in U, |A| < |B|$:}
	\pi_{A} &:= \text{Eine, gemäß Definition aus $U$, zu $A$ passende Bijektion.} \\
	\pi_{B} &:= \text{Eine, gemäß Definition aus $U$, zu $B$ passende Bijektion.} \\
	pos_{BA} &:= \pi_{A} \circ \pi_{B}^{-1} \\
	f(i) &:= \begin{cases}
		\pi_{B}^{-1}(i): & \pi_{B}^{-1}(i) \notin A \\
		f(pos_{BA}(i)): & sonst
	\end{cases} \\
	x &:= f(|A| + 1) \\
	C &:= A \cup \{ x \} \\
	\intertext{$f(|A| + 1)$ findet gemäß dem Schubfachprinzip immer ein $x$ aus $B \backslash A$, da die ersten $|A| + 1$ Elemente von $B$ probiert werden.}
	g(\pi, i) &:= \begin{cases}
		\pi \cup \{ (\pi_{B}^{-1}(i), i) \}: & \pi_{B}^{-1}(i) \notin A  \\
		g(\pi \cup \{ (\pi_{B}^{-1}(i),\ i) \}\ \backslash\ \{ (\pi_{B}^{-1}(i),\ pos_{BA}(i)) \},\ pos_{BA}(i)): & sonst
	\end{cases} \\
	\pi_{C} &:= g(\pi_{A}, |A| + 1)
	\intertext{$g$ erweitert $\pi_{A}$ so, dass $x$ an der Stelle $\pi_{B}(x)$ einsortiert wird. Wenn dort in $\pi_{A}$ bereits ein anderes Element $y$ liegt, wird $y$ nach $\pi_{B}(y)$ geschoben. Sollte dort wiederum ein Element liegen wird es nach dem selben Prinzip weitergeschoben. \\ Gemäß Konstruktion wird $x$ dabei so gewählt, dass $\forall\ \text{geschobenen}\ y: \pi_{B}(y) \le |A| + 1$. Das Schieben findet zwangsläufig ein Ende, sobald ein Element aus $A$ nach $|A| + 1$ geschoben wird, da $\pi_{A}$ dort nichts einsortiert haben kann und somit nichts geschoben werden muss. \\ Da alle Zuordnungen von $\pi_{C}$ entweder aus $\pi_{A}$ oder $\pi_{B}$ stammen, ist $\pi_{C}$ eine Sortierung von $C$, die keine Strafe verursacht.}
	\implies & C \in U
\end{align*}
\qed\

\end{document}
