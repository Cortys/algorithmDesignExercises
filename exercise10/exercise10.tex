\documentclass[12pt,a4paper,bibliography=totocnumbered,listof=totocnumbered]{scrartcl}
\usepackage[ngerman]{babel}
\usepackage[utf8]{inputenc}
\usepackage[T1]{fontenc}

\usepackage{alltt}
\usepackage{upquote}
\usepackage{enumitem}
\usepackage{amsmath}
\usepackage{amsfonts}
\usepackage{amssymb}
\usepackage{pifont}
\usepackage{graphicx}
\usepackage{geometry}
\usepackage{setspace}
\usepackage[right]{eurosym}
\usepackage[printonlyused]{acronym}
\usepackage{lmodern}
\usepackage{blindtext}
\usepackage{soul,xcolor}
\usepackage{listings}
\usepackage{multicol}
\usepackage{syntax}
\usepackage[all,defaultlines=4]{nowidow}
\usepackage{tikz}
\usepackage{marvosym}
\usepackage{mathtools}
\usepackage{booktabs}
\usepackage{algorithm}
\usepackage[noend]{algpseudocode}

\lstset{basicstyle=\footnotesize, captionpos=b, breaklines=true, showstringspaces=false, tabsize=2, frame=lines, numbers=left, numberstyle=\tiny, xleftmargin=2em, framexleftmargin=2em, escapechar=@}

\lstset{literate=%
  {Ö}{{\"O}}1
  {Ä}{{\"A}}1
  {Ü}{{\"U}}1
  {ß}{{\ss}}2
  {ü}{{\"u}}1
  {ä}{{\"a}}1
  {ö}{{\"o}}1
}

\makeatletter
\def\l@lstlisting#1#2{\@dottedtocline{1}{0em}{1em}{\hspace{1,5em} Lst. #1}{#2}}
\makeatother

\geometry{a4paper, top=27mm, left=30mm, right=20mm, bottom=35mm, headsep=10mm, footskip=12mm}

\def\code#1{\texttt{#1}}

\sethlcolor{yellow}

\newcommand*\circled[1]{\tikz[baseline= (char.base)]{
            \node[shape=circle,draw,inner sep=2pt] (char) {#1};}}

\renewcommand{\qed}{\hfill\blacksquare}

\renewcommand{\sfdefault}{ppl}

\renewcommand{\thesubsection}{\alph{subsection}}
\renewcommand*\labelitemi{\blacktriangleright}

% \renewcommand{\syntleft}{\normalfont\itshape}
\renewcommand{\syntright}{}

\makeatletter
\renewcommand*\env@matrix[1][*\c@MaxMatrixCols c]{%
  \hskip -\arraycolsep
  \let\@ifnextchar\new@ifnextchar
  \array{#1}}

\renewcommand\maketitle{
   \begin{center}
     {\LARGE\bfseries Algorithmenentwurf \@title\par\vspace{0.5em}}
     {\scshape\@author\break\@date}
   \end{center}
}
\makeatother

\author{Lukas Brandt: 7011823, Clemens Damke: 7011488, Lukas Giesel: 7011495}


\begin{document}

\title{HA 10}
\date{30. Juni 2016}

\maketitle

\section*{Aufgabe 21}
\label{sec:Aufgabe 21}

\subsection{Algorithmus}
\label{sub:Algorithmus}

\begin{align*}
	in_G(v) :=&\ \{ (v_1, v_{2}) \in E\, |\, v_2 = v  \} \text{ mit $G = (V, E)$} \\
	out_G(v) :=&\ \{ (v_1, v_{2}) \in E\, |\, v_1 = v  \} \text{ mit $G = (V, E)$},\ & w_G(v) := &\ \frac{|in_G(v)|}{|out_G(v)|} \\
	ind(V, E) :=&\ (V,\ \{ e \in E \cap V^{2} \})
\end{align*}

\begin{algorithm}
	\caption{Maximaler azyklischer Teilgraph Approximation}
	\begin{algorithmic}[1]
		\Function{dag}{$G$}
			\State{$(V, E) \gets G$}
			\If{$|V| = 0$}
				\Return{$\emptyset$}
			\EndIf\
			\Statex\
			\State{$s \gets null$}
			\For{$v \in V$}
				\If{$s = null \lor w_G(v) < w_G(s)$}
					$s \gets v$
				\EndIf\
			\EndFor\
			\State{\Return{$out_G(s) \cup \Call{dag}{ind(V \backslash \{ s \}, E)}$}}
		\EndFunction\
	\end{algorithmic}
\end{algorithm}

\subsection{Korrektheit und Laufzeit}
\label{sub:Korrektheit und Laufzeit}

Die Funktionen $in_G$, $out_G$ und $w_G$ lassen sich alle in $\mathcal{O}(|V|)$ berechnen, da jeder Knoten höchstens $|V| - 1$ Nachbarn haben kann. \\
Die Zeilen 1 – 4 sind in $\mathcal{O}(1)$ ausführbar. \\
Die Zeilen 5 – 6 benötigen $\mathcal{O}(|V|^2)$, da $2 |V|$-mal $w_G$ aufgerufen wird. \\
Zeile 7 benötigt $\mathcal{O}(|V| + |V|^3) = \mathcal{O}(|V|^3)$, da es genau $|V| - 1$ rekursive Aufrufe von $dag$ gibt, die wegen den Zeilen 5 – 6 jeweils $\mathcal{O}(|V|^2)$ benötigen.
\\ \\
Insgesamt hat $dag$ also eine Zeitkomplexität von $\mathcal{O}(|V|^3)$.
\\ \\
$dag(G)$ berechnet die Kantenmenge eines DAGs von $G$. \\
Denn $\forall i \in \{ 1, \dots, n \}: \forall (s_i, s_j) \in dag(G): i < j$, wobei $s_1$ das im ersten / direkten Aufruf von $dag$ und $s_n$ das im letzten rekursiven Aufruf selektierte $s$ ist. Es gibt also eine topologische Sortierung von $(V, dag(G))$.

\subsection{Approximationsfaktor}
\label{sub:Approximationsfaktor}

\begin{align*}
	&\frac{\sum_{v \in V} |in_G(v)|}{\sum_{v \in V} |out_G(v)|} = 1 \\
	\implies &\exists v \in V: w_G(v) \le 1 \\
	&\text{Seien } s_1, \dots, s_n \text{ definiert, wie in Aufgabenteil b.} \\
	&\text{Seien } G_1, \dots, G_n \text{ die nach $dag$ übergebenen Graphen, mit $G_1 = G$ und $G_n = (\{ s_n \}, \emptyset)$.} \\
	\implies &\forall i \in \{ 1, \dots, n \}: w_{G_i}(s_i) \le 1 \land in_{G_i}(s_i) = in_G(s_i) \backslash (\cup_{j \in \{ 1, \dots, i-1 \}} out_{G_j}(s_j)) \\
	\implies &\forall i \in \{ 1, \dots, n \}: |out_{G_i}(s_i)| \ge |in_{G_i}(s_i)| \\
	\implies &\forall i \in \{ 1, \dots, n \} \land |out_{G_i}(s_i)| \ge |in_G(s_i)|: \text{min. die Hälfte der Kanten von $s_i$ in $deg(G)$.} \\
	\implies &\forall i \in \{ 1, \dots, n \} \land |out_{G_i}(s_i)| < |in_G(s_i)|: in_G(s_i) \backslash in_{G_i}(s_i) \text{ sind bereits in $deg(G)$.} \\
	\implies &\forall i \in \{ 1, \dots, n \} \land |out_{G_i}(s_i)| < |in_G(s_i)|: |out_{G_i}(s_i) \cup in_G(s_i) \backslash in_{G_i}(s_i)| \ge |in_G(s_i)| \\
	\implies &\forall i \in \{ 1, \dots, n \} \land |out_{G_i}(s_i)| < |in_G(s_i)|: \text{min. die Hälfte der Kanten von $s_i$ in $deg(G)$.} \\
	\implies &\forall i \in \{ 1, \dots, n \}: \text{min. die Hälfte der Kanten von $s_i$ in $deg(G)$.} \\
	\implies &\text{Da $\{ s_1, \dots, s_n \} = V$, ist mindestens die Hälfte der Kanten von $G$ in $deg(G)$.} \\
	\implies &\text{Da $|E| \ge |optDag(G)| \ge |dag(G)| \ge \frac{1}{2} |E|$, hat $dag$ eine Güte von $\frac{1}{2}$.}
\end{align*}
\qed

\end{document}
