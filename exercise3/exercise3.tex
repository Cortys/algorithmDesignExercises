\documentclass[12pt,a4paper,bibliography=totocnumbered,listof=totocnumbered]{scrartcl}
\usepackage[ngerman]{babel}
\usepackage[utf8]{inputenc}
\usepackage[T1]{fontenc}

\usepackage{alltt}
\usepackage{upquote}
\usepackage{enumitem}
\usepackage{amsmath}
\usepackage{amsfonts}
\usepackage{amssymb}
\usepackage{pifont}
\usepackage{graphicx}
\usepackage{geometry}
\usepackage{setspace}
\usepackage[right]{eurosym}
\usepackage[printonlyused]{acronym}
\usepackage{lmodern}
\usepackage{blindtext}
\usepackage{soul,xcolor}
\usepackage{listings}
\usepackage{multicol}
\usepackage{syntax}
\usepackage[all,defaultlines=4]{nowidow}
\usepackage{tikz}
\usepackage{marvosym}
\usepackage{mathtools}
\usepackage{booktabs}
\usepackage{algorithm}
\usepackage[noend]{algpseudocode}

\lstset{basicstyle=\footnotesize, captionpos=b, breaklines=true, showstringspaces=false, tabsize=2, frame=lines, numbers=left, numberstyle=\tiny, xleftmargin=2em, framexleftmargin=2em, escapechar=@}

\lstset{literate=%
  {Ö}{{\"O}}1
  {Ä}{{\"A}}1
  {Ü}{{\"U}}1
  {ß}{{\ss}}2
  {ü}{{\"u}}1
  {ä}{{\"a}}1
  {ö}{{\"o}}1
}

\makeatletter
\def\l@lstlisting#1#2{\@dottedtocline{1}{0em}{1em}{\hspace{1,5em} Lst. #1}{#2}}
\makeatother

\geometry{a4paper, top=27mm, left=30mm, right=20mm, bottom=35mm, headsep=10mm, footskip=12mm}

\def\code#1{\texttt{#1}}

\sethlcolor{yellow}

\newcommand*\circled[1]{\tikz[baseline= (char.base)]{
            \node[shape=circle,draw,inner sep=2pt] (char) {#1};}}

\renewcommand{\qed}{\hfill\blacksquare}

\renewcommand{\sfdefault}{ppl}

\renewcommand{\thesubsection}{\alph{subsection}}
\renewcommand*\labelitemi{\blacktriangleright}

% \renewcommand{\syntleft}{\normalfont\itshape}
\renewcommand{\syntright}{}

\makeatletter
\renewcommand*\env@matrix[1][*\c@MaxMatrixCols c]{%
  \hskip -\arraycolsep
  \let\@ifnextchar\new@ifnextchar
  \array{#1}}

\renewcommand\maketitle{
   \begin{center}
     {\LARGE\bfseries Algorithmenentwurf \@title\par\vspace{0.5em}}
     {\scshape\@author\break\@date}
   \end{center}
}
\makeatother

\author{Lukas Brandt: 7011823, Clemens Damke: 7011488, Lukas Giesel: 7011495}


\begin{document}

\title{HA 3}
\date{28. April 2016}

\maketitle

\section*{Aufgabe 5}
\label{sec:Aufgabe 5}

\subsection{Beweis}
\label{sub:Beweis}

\begin{align*}
	P  =&\ (A_{1 1} + A_{2 2}) \cdot (B_{1 1} + B_{2 2})
	  &=&\ A_{1 1} \cdot B_{1 1} + A_{1 1} \cdot B_{2 2} + A_{2 2} \cdot B_{1 1} + A_{2 2} \cdot B_{2 2} \\
	Q  =&\ (A_{2 1} + A_{2 2}) \cdot B_{1 1}
	  &=&\ A_{2 1} \cdot B_{1 1} + A_{2 2} \cdot B_{1 1} \\
	R  =&\ A_{1 1} \cdot (B_{1 2} - B_{2 2})
	  &=&\ A_{1 1} \cdot B_{1 2} - A_{1 1} \cdot B_{2 2} \\
	S  =&\ A_{2 2} \cdot (B_{2 1} - B_{1 1})
	  &=&\ A_{2 2} \cdot B_{2 1} - A_{2 2} \cdot B_{1 1} \\
	T  =&\ (A_{1 1} + A_{1 2}) \cdot B_{2 2}
	  &=&\ A_{1 1} \cdot B_{2 2} + A_{1 2} \cdot B_{2 2} \\
	U  =&\ (A_{2 1} - A_{1 1}) \cdot (B_{1 1} + B_{1 2})
	  &=&\ A_{2 1} \cdot B_{1 1} + A_{2 1} \cdot B_{1 2} - A_{1 1} \cdot B_{1 1} - A_{1 1} \cdot B_{1 2} \\
	V  =&\ (A_{1 2} - A_{2 2}) \cdot (B_{2 1} + B_{2 2})
	  &=&\ A_{1 2} \cdot B_{2 1} + A_{1 2} \cdot B_{2 2} - A_{2 2} \cdot B_{2 1} - A_{2 2} \cdot B_{2 2} \\ \\
	%
	C_{1 1} =&\ P + S - T + V
	       &=&\ A_{1 1} \cdot B_{1 1} \hlc{+ A_{1 1} \cdot B_{2 2}} \hlc[lime]{+ A_{2 2} \cdot B_{1 1}} \hlc[orange]{+ A_{2 2} \cdot B_{2 2}} \\
		  &&+&\ \hlc[magenta]{A_{2 2} \cdot B_{2 1}} \hlc[lime]{- A_{2 2} \cdot B_{1 1}} \hlc{- A_{1 1} \cdot B_{2 2}} \hlc[cyan]{- A_{1 2} \cdot B_{2 2}} \\
	      &&+&\ A_{1 2} \cdot B_{2 1} \hlc[cyan]{+ A_{1 2} \cdot B_{2 2}} \hlc[magenta]{- A_{2 2} \cdot B_{2 1}} \hlc[orange]{- A_{2 2} \cdot B_{2 2}} \\
		  &&=&\ A_{1 1} \cdot B_{1 1} + A_{1 2} \cdot B_{2 1}\ \checkmark \\
	C_{1 2} =&\ R + T
	       &=&\ A_{1 1} \cdot B_{1 2} \hlc{- A_{1 1} \cdot B_{2 2} + A_{1 1} \cdot B_{2 2}} + A_{1 2} \cdot B_{2 2} \\
	      &&=&\ A_{1 1} \cdot B_{1 2} + A_{1 2} \cdot B_{2 2}\ \checkmark \\
	C_{2 1} =&\ Q + S
		   &=&\ A_{2 1} \cdot B_{1 1} \hlc{+ A_{2 2} \cdot B_{1 1}} + A_{2 2} \cdot B_{2 1} \hlc{- A_{2 2} \cdot B_{1 1}} \\
	      &&=&\ A_{2 1} \cdot B_{1 1} + A_{2 2} \cdot B_{2 1}\ \checkmark \\
	C_{2 2} =&\ P + R - Q + U
		   &=&\ \hlc{A_{1 1} \cdot B_{1 1}} \hlc[lime]{+ A_{1 1} \cdot B_{2 2}} \hlc[orange]{+ A_{2 2} \cdot B_{1 1}} + A_{2 2} \cdot B_{2 2} \\
		  &&+&\ \hlc[magenta]{A_{1 1} \cdot B_{1 2}} \hlc[lime]{- A_{1 1} \cdot B_{2 2}} \hlc[cyan]{- A_{2 1} \cdot B_{1 1}} \hlc[orange]{- A_{2 2} \cdot B_{1 1}} \\
		  &&+&\ \hlc[cyan]{A_{2 1} \cdot B_{1 1}} + A_{2 1} \cdot B_{1 2} \hlc{- A_{1 1} \cdot B_{1 1}} \hlc[magenta]{- A_{1 1} \cdot B_{1 2}} \\
	      &&=&\ A_{2 1} \cdot B_{1 2} + A_{2 2} \cdot B_{2 2}\ \checkmark
\end{align*}
\\
$C_{1 1}$, $C_{1 2}$, $C_{2 1}$, $C_{2 2}$ sind äquivalent zur Definition in \emph{Aufgabe 3} und ergeben zusammengesetzt somit $C$. $\qed$
\\ \\
Die Anzahl der benötigten Multiplikationen beträgt 7, da in jeder der sieben Hilfsvariablen $P$, $Q$, $R$, $S$, $T$, $U$, $V$ jeweils eine Multiplkation stattfindet und sonst nirgends.\\ \\
Die Anzahl der benötigten Additionen beträgt $10 + 8 = 18$, da zur Berechnung der Hilfsvariablen jeweils 2, 1, 1, 1, 1, 2, 2 ($\sum = 10$) und zur Berechnung von $C_{i j}$ jeweils 3, 1, 1, 3 ($\sum = 8$) Additionen benötigt werden.

\subsection{Algorithmus}
\label{sub:Algorithmus}

\begin{algorithm}
	\caption{Matrizenmultiplikation}
	\begin{algorithmic}[1]
		\Require $\exists k \in \mathbb{N}_0, n := 2^k:
		A = \begin{bmatrix} a_{1 1} & \cdots & a_1_n \\ \vdots & \ddots & \vdots \\ a_n_1 & \cdots & a_n_n \end{bmatrix} \land
		B = \begin{bmatrix} b_{1 1} & \cdots & b_1_n \\ \vdots & \ddots & \vdots \\ b_n_1 & \cdots & b_n_n \end{bmatrix}$
		\Statex
		\Function{M-Mult}{$A$, $B$}
			\If{$n = 1$}
				\State \Return $a_{1 1} \cdot b_{1 1}$ \Comment $\mathcal{O}(1)$
			\EndIf
			\Statex
			\State $A_{1 1}, A_{1 2}, A_{2 1}, A_{2 2} \gets (\frac{n}{2} \times \frac{n}{2})\text{-Submatrizen von A}$ \Comment $\mathcal{O}(n^2)$
			\State $B_{1 1}, B_{1 2}, B_{2 1}, B_{2 2} \gets (\frac{n}{2} \times \frac{n}{2})\text{-Submatrizen von B}$ \Comment $\mathcal{O}(n^2)$
			\Statex
			\State $P, Q, R, S, T, U, V \gets$\ definiert wie in Aufgabenstellung, wobei '$\cdot$' durch \mbox{\Call{M-Mult}{}} und '$+$' durch \Call{M-Add}{} ersetzt wird. $\implies 7 \times \Call{M-Mult}{} = \mathcal{O}(7 \cdot T(\frac{n}{2})), 10 \times \Call{M-Add}{} = \mathcal{O}(n^2)$
			\Statex
			\State $C_{1 1}, C_{1 2}, C_{2 1}, C_{2 2} \gets$\ definiert wie in Aufgabenstellung, wobei '$+$' durch \Call{M-Add}{} ersetzt wird. $\implies 8 \times \Call{M-Add}{} = \mathcal{O}(n^2)$
			\Statex
			\State \Return $\begin{bmatrix}
				C_{1 1} & C_{1 2} \\
				C_{2 1} & C_{2 2}
			\end{bmatrix}$ \Comment $\mathcal{O}(n^2)$
		\EndFunction
		\Statex
		\Function{M-Add}{$A$, $B$}
			\ForAll{$i, j \in \{1, \ldots, n\}$}
				\State $c_i_j \gets a_i_j + b_i_j$
			\EndFor
			\Statex
			\State \Return $\begin{bmatrix} c_{1 1} & \cdots & c_1_n \\ \vdots & \ddots & \vdots \\ c_n_1 & \cdots & c_n_n \end{bmatrix}$
		\EndFunction
	\end{algorithmic}
\end{algorithm}

\textbf{Laufzeit:}
\begin{align*}
	T(n) &= 7c \cdot T(\frac{n}{2}) + c \cdot n^2 \\
	\underset{\mathrm{Mastertheorem}\ a > b}{\implies} T(n) &\le \frac{7}{7 - 2}c \cdot n^{\log_2{7}} = \frac{7}{5}c \cdot n^{\log_2{7}} = \mathcal{O}(n^{\log_2{7}}) = \mathcal{O}(n^{2.8074})
\end{align*}

\pagebreak

\setcounter{subsection}{0}

\section*{Aufgabe 6}
\label{sec:Aufgabe 6}

\subsection{Beweis}
\label{sub:Beweis}

\emph{z.z. } $A \Leftrightarrow B$ mit $A, B$ linke bzw. rechte Aussage aus Aufgabenstellung.

\begin{align*}
	\Gamma(v, G) :=&\ \text{Nachbarn des Knotens $v$ im Graphen $G$} \\
	pos(a) :=&\ i\ \text{sodass}\ a = v_i
\end{align*}

\begin{itemize}
	\item $A \Rightarrow B$:
		\begin{align*}
			\intertext{Finde unter den ersten $d_0$ Knoten (gemäß einer Sortierung der Knoten nach ihren Graden) die, die nicht Nachbarn von $v_0$ sind ($=Y$) und welche Knoten stattdessen Nachbarn von $v_0$ sind ($=X$):}
			X :=&\ \{ k \in \Gamma(v_0, G)\ |\ pos(k) > d_0 \} = \{ x_1, \dots, x_{|X|} \} \\
			Y :=&\ \{ k \notin \Gamma(v_0, G)\ |\ pos(k) \le d_0 \} = \{ y_1, \dots, y_{|Y|} \} \\
			\implies&\ |X| = |Y| =: z \\\\
			\sigma(y_i) :=&\ v\ \text{sodass}\ \{ y_i, v \} \in E \land v \neq x_i
			\intertext{$\sigma$ ist für alle $y_i$ definiert, da:}
			\forall y_i \in Y:&\ deg(y_i) = d_{pos(y_i)} \ge d_{pos(x_i)} = deg(x_i) \ge 1 \text{, da $x_i$ Nachbar von $v_0$ ist.} \\
			\implies&\ \begin{cases}
				deg(y_i) = 1 \Rightarrow deg(x_i) = 1 \Rightarrow \text{Nachbar von $y_i$ ist nicht $x_i$, da $\Gamma(x_i, G) = \{ v_0 \}$.} \\
				deg(y_i) > 1 \Rightarrow \exists\ \text{Nachbar von $y_i$, der nicht $x_i$ ist.}
			\end{cases}
			\intertext{Entferne nun aus $E$ alle Nachbarn von $v_0$. Der resultierende Graph entspricht dem induzierten Teilgraphen von $G$ ohne $v_0$:}
			E_0 :=&\ E \setminus \{ \{ v_0, v \}\ |\ v \in V \}
			\intertext{$E_0$ erfüllt $B$ nicht notwendigerweise, da $\forall x \in X$ der Grad von Knoten ‘hinter' den ersten $d_0$ verringert wurde und dafür der Grad von Knoten 'vor' den ersten $d_0$ ($\in Y$) unverändert blieb. Um dies zu korrigieren wird von jedem Knoten aus $Y$ eine Kante entfernt und stattdessen einem Knoten in $X$ hinzugefügt:}
			E' :=&\ E_0 \setminus \{ \{ y_i, \sigma(y_i) \}\ |\ i \in \{ 1, \dots, z \} \} \cup \{ \{ x_i, \sigma(y_i) \}\ |\ i \in \{ 1, \dots, z \} \} \\
			G' :=&\ (\{ v_1, \dots, v_{n-1} \}, E') \\
			\implies&\ \text{$G'$ erfüllt $B$ gemäß Konstruktion.}\ \checkmark
		\end{align*}

	\item $A \Leftarrow B$:
		\begin{align*}
			E :=&\ E' \cup \{ \{ v_0, v_i \}\ |\ i \in \{ 1, \dots, d_0 \} \} \\
			G :=&\ (\{ v_0, \dots, v_{n-1} \}, E) \\
			\implies&\ \text{$G$ erfüllt $A$ gemäß Konstruktion, da der Grad der ersten $d_0$ Knoten um 1} \\
			&\ \text{erhöht wurde, $v_0$ genau $d_0$ Kanten hat und somit $\forall v_i: deg(v_i) = d_i$ gilt.}\ \checkmark
		\end{align*}
\end{itemize}

\qed

\subsection{Algorithmus}
\label{sub:Algorithmus}

\begin{algorithm}
	\caption{Graphexistenz}
	\begin{algorithmic}[1]
		\Require $d_0 \ge \dots \ge d_n, d_i \in \mathbb{N}_0$
		\Statex
		\Function{GraphExists}{$d_0, \dots, d_n$}
			\If{$d_0 < 0$} \Return false \EndIf
			\If{$d_0 = 0$} \Return true \EndIf
			\Statex

			\For{$i \gets 0$ to $n - 1$}
				\For{$j \gets i + 1$ to $i + d_i$}
					\If{$d_j = 0$} \Return false \EndIf
					\State $d_j \gets d_j - 1$
				\EndFor
				\Statex
				\If{$i + d_i < n \land d_{i + d_i} < d_{i + d_i + 1}$} \Return false \EndIf
			\EndFor
			\Statex
			\State \Return true
		\EndFunction
	\end{algorithmic}
\end{algorithm}

\subsection{Laufzeit}
\label{sub:Laufzeit}

Die Zeilen 2, 3, 9 benötigen konstante Zeit. Die Zeilen 6, 7 werden $\mathcal{O}(n^2)$ mal ausgeführt. Zeile 8 $\mathcal{O}(n)$ mal.\\\\
$\implies T(n) = \mathcal{O}(n^2)$

\end{document}
